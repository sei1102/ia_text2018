\documentclass[11pt,a4paper]{jarticle}
\usepackage[dvipdfmx]{graphicx}
\usepackage{url}

\renewcommand{\baselinestretch}{1.05} 
\marginparwidth=0cm
\topmargin=-1cm
\headheight=0.3cm
\headsep=0.7cm
\oddsidemargin=0cm
\evensidemargin=0cm
%\textwidth=43zw
\textwidth=15.92cm
%\textheight=43.3\baselineskip
\baselineskip = 0.5744cm
\textheight=43\baselineskip

\itemsep=0.05\baselineskip
\parsep=0pt
\topsep=0.01\baselineskip
\partopsep=0pt
\listparindent=0zw

%% header and footer
\usepackage{fancyhdr}
\pagestyle{fancy}
\lhead{2017年度 春学期授業}
\chead{インタラクティブ・アート実習}
\rhead{担当教員: 松下 光範}
\cfoot{\thepage}
\renewcommand{\headrulewidth}{0pt}
\renewcommand{\footrulewidth}{0pt}

\usepackage{ascmac}
\usepackage{listings,jlisting}
\usepackage{color}
\definecolor{OliveGreen}{cmyk}{0.64,0,0.95,0.40}
\definecolor{colFunc}{rgb}{1,0.07,0.54}
\definecolor{CadetBlue}{cmyk}{0.62,0.57,0.23,0}
\definecolor{Brown}{cmyk}{0,0.81,1,0.60}
\definecolor{colID}{rgb}{0.63,0.44,0}
\definecolor{rulesepcolor}{gray}{0.666}
\lstset{
  language=Java,%プログラミング言語によって変える。
  basicstyle={\ttfamily\small},
  keywordstyle={\color{OliveGreen}},
  %[2][3]はプログラミング言語によってあったり、なかったり
  keywordstyle={[2]\color{colFunc}},
  keywordstyle={[3]\color{CadetBlue}},%
  commentstyle={\color{Brown}},
  %identifierstyle={\color{colID}},
  stringstyle=\color{blue},
  tabsize=2,
  %frame=trBL,
  %numbers=left,
  numberstyle={\ttfamily\small},
  breaklines=true,%折り返し
  %backgroundcolor={\color[gray]{.95}},
  framexleftmargin=0mm,
  frame=single,
  rulesepcolor=\color{rulesepcolor},
  captionpos=b
}


%%%%%%%%%%%%%%%%%%%%%%%%%%%%%%%%%%%%%%%%%%%%%%%%%%%%%%%%%%%%%%%%
\begin{document}
\section*{付録}
\subsection*{ArduinoのInput/Outputまとめ}

\subsubsection*{Digital Input/Output}
\begin{lstlisting}
 import processing.serial.*;
 import cc.arduino.*;

 Arduino arduino;

 int inputPin  = 0;
 int outputPin = 1;

 void setup() {
   // COMポートの指定は各々異なるので、
   // デバイスマネージャなどで調べて変更する
   arduino = new Arduino(this,  "/dev/tty.usbmodem1411");

   // inputPinをInputに
   ardino.pinMode(inputPin,  Arduino.INPUT);

   // outputPinをOutputに 
   ardino.pinMode(outputPin, Arduino.OUTPUT);
 }

 void draw() {
   // Digital Input
   // 戻り値: Arduino.HIGH or Arduino.LOW
   int input = arduino.digitalRead(inputPin);

   // Digital Output
   // outputPinをHIGH(5V)に
   arduino.digitalWrite(outputPin, Arduino.HIGH);
   // outputPinをLOW(0V)に
   arduino.digitalWrite(outputPin, Arduino.LOW);
 } 
\end{lstlisting}

\newpage

\subsubsection*{Analog Input/Output}
\begin{lstlisting}
 import processing.serial.*;
 import cc.arduino.*;

 Arduino arduino;

 int inputPin  = 0; // A0を使うときは0に
 int outputPin = 9; // PWMが使えるピンを選ぶ 

 void setup() {
   arduino = new Arduino(this,  "/dev/tty.usbmodem1411");

   // Analog Input を使うときは Pin Mode の設定は必要ない
   // ardino.pinMode(inputPin,  Arduino.INPUT);

   // outputPinをOutputに 
   ardino.pinMode(outputPin, Arduino.OUTPUT);
 }

 void draw() {
   // Analog Input
   // 戻り値: 0〜1023の整数
   int input = arduino.analogRead(inputPin);

   // Analog Output
   // Analogなので、outputする値を0〜255で指定
   arduino.analogWrite(outputPin, 0);
   arduino.analogWrite(outputPin, 127);
   arduino.analogWrite(outputPin, 255);
 } 
\end{lstlisting}



\end{document}